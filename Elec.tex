\section{Электроника}

\subsection{Полупроводниковая электроника}

\subsection{Образование энергетических зон в твердом теле. Одномерная модель. Разрешенная и запрещенная зона. Зона проводимости, валентная зона.}

\subsection{Статистика носителей заряда в твердом теле. Формула Ферми-Дирака. Уровень Ферми. Формула Максвелла-Больцмана для невырожденного полупроводника.

\subsection{Положение уровня Ферми в зависимости от температуры}

\subsection{Электропроводимость собственных полупроводников. Примесные полупроводники. Полупроводники p-типа и n-типа. Зонная диаграмма для примесных полупроводников. Энергия активации доноров, энергия активации акцепторов.}

\subsection{Температурная зависимости электропроводимости полупроводников с точки зрения зонной теории.}

\subsection{Термоэлектронная эмиссия. Понятие о работе выхода, электронном средстве и контактной разности потенциалов.}
\subsection{Контакт металл-полупроводник. Барьер Шоттки.}
\subsection{Контакт полупроводник-полупроводник. Образование и энергетическая диаграмма электронно-дырочного перехода (ЭДП).}
\subsection{Равновесное состояние ЭДП. ВЫсота потенциального барьера и контактная разность потенциалов ЭДП. \subsection{Распределение напряженности потенциала электрического поля в ЭДП.}
\subsection{Соотношения для расчета толщины ЭДП. Неравновесное состояние ЭДП. ВАХ перехода.}
\subsection{Влияние процессов генерации и рекомбинации на ВАХ диода. Влияние толщины базы диода на его вольт-амперную характеристику.}
\subsection{Частотные и импульсные свойства полупроводникового диода.}
\subsection{Барьерная емкость контакта. Диффузная емкость. Эквивалентная схема диода.}
\subsection{Пробой ЭДП. Тепловой пробой. Напряжение пробоя.}
\subsection{Лавинный пробой ЭДП. Критерий развития лавинного пробоя ЭДП. Выражения для обратной вольт-амперной характеристики при развитии лавинного пробоя.}
\subsection{Туннельный пробой. Туннельный и обращенный диоды.}
Принцип действия транзистора в качестве усилителя. Основные характеристики транзистора: коэффициент передачи тока, эффективность эмиттера, коэффициент переноса.}
\subsection{Зависимость коэффициента инжекции (эффективности эмиттера) и коэффициента передачи от характеристик для включений по схеме с общей базой.}
\subsection{Уравнения Эберса-Молла. Вывод выражений входных и выходных характеристик для включений по схеме с общим эмиттером.}
\subsection{Частотная зависимость коэффициента передачи тока в схеме с общей базой, общим эмиттером.}
\subsection{Эквивалентная схема транзистора.}
