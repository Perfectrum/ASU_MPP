%\begin{document}

\section{Программирование}


\subsection{Численные методы}

\subsubsection{Аппроксимация}
Известны данные $(t_{i},b_{i});i=1,...,m$
$$b(t)-?$$
Рассмотрим линейную модель (комбинация модельных функций линейна)
$$b_{i}=x_{1}\phi_{1}(t_{i})+x_{2}\phi_{2}(t_{i})+...+x_{n}\phi_{n}(t_{i})$$
$\phi_{i}(t)-$модельные функции(задаем сами)\\
$x_{i}-$параметры модели\\\\
Введем матрицу $A$ $m\times n$, где $m$-число данных, $n$-количество модельных функций
$$a_{ij}=\left(A\right)_{ij}=\phi_{j}(t_{i})$$
\begin{equation*}
A = \left(
\begin{array}{cccc}
\phi_{1}(t_{1}) & \phi_{2}(t_{1}) & \ldots & \phi_{n}(t_{1})\\
\phi_{1}(t_{2}) & \phi_{2}(t_{2}) & \ldots & \phi_{n}(t_{2})\\
\vdots & \vdots & \ddots & \vdots\\
\phi_{1}(t_{m}) & \phi_{2}(t_{m}) & \ldots & \phi_{n}(t_{m})
\end{array}
\right)
\end{equation*}
В таком случае задачу можно переписать в виде:
$$b\approx Ax\Longrightarrow b-Ax\approx 0$$
$r=b-Ax$ --- вектор невязок\\
$$r\longrightarrow 0\Leftrightarrow \min_{x}\sum_{j=1}^{m}[(b-Ax)_{j}]^{2}=\min_{x}||b-Ax||_{2}^{2}=(b-Ax)^{T}(b-Ax)$$
Можно ввести веса $\omega_{j}:$ $$\min_{x}\sum_{j=1}^{m}[\omega_{j}(b-Ax)_{j}]^{2}$$
где $\omega_{j}=\frac{1}{\epsilon_{j}}$,$\epsilon_{j}$ --- ошибка измерения в $j$-ой точке\\\\
\textbf{Нахождение параметров модели:}
\begin{enumerate}
    \item Нормальные уравнения
    $$r^{2}=(b-Ax)^{T}(b-Ax)=b^{T}b-2b^{T}Ax+x^{T}A^{T}Ax\longrightarrow min$$
    $$\frac{d(r^{2})}{dx}=0\Longrightarrow (A^{T}A)x=(A^{T}b)$$
    \item Ортоганальные факторизации\\
    Ортоганальное преобразование --- домножение на $P$: $P^{T}P=1$
    \item Преобразование Хаусхолдера
    $$P=1-2\frac{\mathcal{V}\mathcal{V}^{T}}{\mathcal{V}^{T}\mathcal{V}}$$
    Позволяет привести матрицу $A$ к верхнетреугольному виду, сохраняя норму.\\
    Пусть $a$ --- произвольный вектор-столбец, тогда:
    \begin{equation*}
    a = \left(
    \begin{array}{cccc}
    a_{1}\\
    a_{2}\\
    \vdots\\
    a_{m}
    \end{array}
    \right)
    \end{equation*}
    \begin{equation*}
    Pa = \left(
    \begin{array}{cccc}
    \alpha\\
    0\\
    \vdots\\
    0
    \end{array}
    \right)
    \end{equation*}
    $\alpha=||a||_{2}$\\\\
    Пример:
\begin{equation*}
A = \left(
\begin{array}{cccc}
\times & \times & \times \\
\times & \times & \times \\
\times & \times & \times \\
\times & \times & \times
\end{array}
\right)
\end{equation*}
\begin{equation*}
PA = \left(
\begin{array}{cccc}
\alpha & \times & \times \\
0 & \times & \times \\
0 & \times & \times \\
0 & \times & \times
\end{array}
\right)
\end{equation*}
\begin{equation*}
P_{1}PA = \left(
\begin{array}{cccc}
1 & 0 \\
0 & \overline{P}
\end{array}
\right)
\left(
\begin{array}{cccc}
\alpha & \times & \times \\
0 & \overline{\alpha} & \times \\
0 & 0 & \times \\
0 & 0 & \times
\end{array}
\right)
\end{equation*}
\begin{equation*}
P_{2}P_{1}PA = \left(
\begin{array}{cccc}
1 & 0 & 0\\
0 & 1 & 0\\
0 & 0 & \widetilde{P}
\end{array}
\right)
\left(
\begin{array}{cccc}
\alpha & \times & \times \\
0 & \overline{\alpha} & \times \\
0 & 0 & \widetilde{\alpha} \\
0 & 0 & 0
\end{array}
\right)
\end{equation*}
Формула для $\mathcal{V}$:\\
$$Pa=\alpha(1,0,...,0)^{T}=\alpha e_{1}\Longrightarrow Pa=\left(P=1-2\frac{\mathcal{V}\mathcal{V}^{T}}{\mathcal{V}^{T}\mathcal{V}}\right)a=\alpha e_{1}$$
Отсюда:
$$\mathcal{V}=a\pm ||a||_{2}e_{1}$$
    \end{enumerate}

%\end{document}